\documentclass[main=english]{ufsc-thesis-rn46-2019}

\usepackage[T1]{fontenc} % fontes
\usepackage[utf8]{inputenc} % UTF-8
\usepackage{lipsum} % Gerador de texto
\usepackage{pdfpages} % Inclui PDF externo (ficha catalográfica)

%%%%%%%%%%%%%%%%%%%%%%%%%%%%%%%%%%%%%%%%%%%%%%%%%%%%%%%%%%%%%%%%%%%%
%%% Configurações da classe (dados do trabalho)                  %%%
%%%%%%%%%%%%%%%%%%%%%%%%%%%%%%%%%%%%%%%%%%%%%%%%%%%%%%%%%%%%%%%%%%%%

% Preâmbulo
\titulo{Constant time root finding techniques in binary finite fields}
\autor{Douglas Marcelino Beppler Martins}
% Importante! Para documentos em inglês, não use today, digite a data em
% pt_BR, como deve aparecer na folha de certificação.
\data{1 de Agosto de 2019}
\instituicao{Universidade Federal de Santa Catarina}
\centro{Centro Tecnológico}
\programa{Programa de Pós-Graduação em Ciência da Computação}
\dissertacao
\local{Florianópolis}
\titulode{Mestre em Ciência da Computação}

% Orientador, coorientador, membros da banca e coordenador
% As regras da BU agora exigem que Dr. apareça **depois** do nome
\orientador{Prof. Ricardo Felipe Custódio, Dr.}
% \coorientador{Prof. Lars Thørväld, Dr.}
\membrobanca{Prof. Valerie Béranger, Dr.}{Universidade Federal de Santa Catarina}
\membrobanca{Prof. Mordecai Malignatus, Dr.}{Universidade Federal de Santa Catarina}
\membrobanca{Prof. Huifen Chan, Dr.}{Universidade Federal de Santa Catarina}
% Dica: se feminino, \coordenadora
\coordenador{Prof. Charles Palmer, Dr}

\begin{document}

%%%%%%%%%%%%%%%%%%%%%%%%%%%%%%%%%%%%%%%%%%%%%%%%%%%%%%%%%%%%%%%%%%%%
%%% Principais elementos pré-textuais                            %%%
%%%%%%%%%%%%%%%%%%%%%%%%%%%%%%%%%%%%%%%%%%%%%%%%%%%%%%%%%%%%%%%%%%%%

% Inicia parte pré-textual do documento capa, folha de rosto, folha de
% aprovação, aprovação, resumo, lista de tabelas, lista de figuras, etc.
\pretextual%
\imprimircapa%
\imprimirfolhaderosto*
\protect\incluirfichacatalografica{ficha.pdf}
\imprimirfolhadecertificacao

\begin{dedicatoria}
  Este trabalho é dedicado à wikipedia e ao stackoverflow. 
\end{dedicatoria}

\begin{agradecimentos}
  O presente trabalho foi realizado com apoio da Coordenação de Aperfeiçoamento de Pessoal de Nível Superior -- Brasil (CAPES) -- Código de Financiamento 001.
\end{agradecimentos}

\begin{epigrafe}
  For a number of years I have been familiar with the observation that the quality of programmers is a decreasing function of the density of go to statements in the programs they produce \\
  \cite{dijkstra1968}
\end{epigrafe}


\begin{resumo}[Resumo]
  Aqui deve ser inserido um resumo de 150 a 500 palavras (limitação de tamanho dada pela BU). A linguagem deve ser português e a hifenização já foi alterada. O resumo em português deve preceder o resumo em inglês, mesmo que o trabalho seja escrito em inglês. A BU também diz que deve ser usada a voz ativa e o discurso deve ser na 3ª pessoa. A estrutura do resumo pode ser similar a estrutura usada em artigos: Contexto -- Problema -- Estado da arte -- Solução proposta  -- Resultados.

  % Atenção! a BU exige separação através de ponto (.). Ela recomanda de 3 a 5 keywords
  \vspace{\baselineskip} 
  \textbf{Palavras-chave:} Palavra-chave. Ponto como separador. Bla.
\end{resumo}


\begin{resumo}[Resumo Estendido]
  \section*{Introdução} 
  A hifenização é alterada para \texttt{brazil}, mesmo para documentos em inglês. Descrever brevemente esses itens exigidos pela BU. Como a RN 95/CUn/2017 é mais recente e impõe outras regras a revelia de regimentos e regulamentos, é mais sábio obedecê-la. Lembre que esse resumo estendido deve term entre 2 e 5 páginas.
  
  \lipsum[1]
  \section*{Objetivos} 
  \lipsum[2]
  \section*{Metodologia} 
  \lipsum[3]
  \section*{Resultados e Discussão} 
  \lipsum[4]
  \section*{Considerações Finais} 
  \lipsum[5]

  \vspace{\baselineskip}  % Atenção! manter igual ao resumo
  \textbf{Palavras-chave:} Palavra-chave. Outra Palavra-chave composta. Bla.
\end{resumo}


\begin{abstract}
  Enlish version of the plain ``resumo'' above. Done with environment
  \texttt{abstract}. Hyphenization is automatically changed to english.

  \vspace{\baselineskip} 
  \textbf{Keywords:} Keyword. Another Compound Keyword. Bla.
\end{abstract}

\listoffigures*

\begin{listadesimbolos}

\end{listadesimbolos}

\tableofcontents*%

%%%%%%%%%%%%%%%%%%%%%%%%%%%%%%%%%%%%%%%%%%%%%%%%%%%%%%%%%%%%%%%%%%%%
%%% Corpo do texto                                               %%%
%%%%%%%%%%%%%%%%%%%%%%%%%%%%%%%%%%%%%%%%%%%%%%%%%%%%%%%%%%%%%%%%%%%%
\textual%

\chapter{Introdução}
\label{ch:intro}


\section{Autores, suporte e atualizações}


%%%%%%%%%%%%%%%%%%%%%%%%%%%%%%%%%%%%%%%%%%%%%%%%%%%%%%%%%%%%%%%%%%%%
%%% Elementos pós-textuais                                       %%%
%%%%%%%%%%%%%%%%%%%%%%%%%%%%%%%%%%%%%%%%%%%%%%%%%%%%%%%%%%%%%%%%%%%%

\postextual
\bibliography{example}

\end{document}
