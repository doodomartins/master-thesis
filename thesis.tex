\documentclass[brazil, english]{ufsc-thesis}

\usepackage[T1]{fontenc} % fontes
\usepackage[utf8]{inputenc} % UTF-8
\usepackage{lipsum} % Gerador de texto
\usepackage{pdfpages} % Inclui PDF externo (ficha catalográfica)

%%%%%%%%%%%%%%%%%%%%%%%%%%%%%%%%%%%%%%%%%%%%%%%%%%%%%%%%%%%%%%%%%%%%
%%% Configurações da classe (dados do trabalho)                  %%%
%%%%%%%%%%%%%%%%%%%%%%%%%%%%%%%%%%%%%%%%%%%%%%%%%%%%%%%%%%%%%%%%%%%%
\titulo{Constant time root finding techniques in binary finite fields}
\autor{Douglas Marcelino Beppler Martins}
\data{1 de Agosto de 2019}
\instituicao{Universidade Federal de Santa Catarina}
\centro{Centro Tecnológico}
\programa{Programa de Pós-Graduação em Ciência da Computação}
\dissertacao
\local{Florianópolis}
\titulode{Mestre em Ciência da Computação}

% Orientador, coorientador, membros da banca e coordenador
% As regras da BU agora exigem que Dr. apareça **depois** do nome
\orientador{Prof. Ricardo Felipe Custódio, Dr.}
\membrobanca{Prof. Valerie Béranger, Dr.}{Universidade Federal de Santa Catarina}
\membrobanca{Prof. Mordecai Malignatus, Dr.}{Universidade Federal de Santa Catarina}
\membrobanca{Prof. Huifen Chan, Dr.}{Universidade Federal de Santa Catarina}
\coordenadora{Prof. Vania Bogorny, Dr}

\begin{document}
%%%%%%%%%%%%%%%%%%%%%%%%%%%%%%%%%%%%%%%%%%%%%%%%%%%%%%%%%%%%%%%%%%%%
%%% Principais elementos pré-textuais                            %%%
%%%%%%%%%%%%%%%%%%%%%%%%%%%%%%%%%%%%%%%%%%%%%%%%%%%%%%%%%%%%%%%%%%%%

% Inicia parte pré-textual do documento capa, folha de rosto, folha de
% aprovação, aprovação, resumo, lista de tabelas, lista de figuras, etc.
\pretextual%
\imprimircapa%
\imprimirfolhaderosto*
\protect\incluirfichacatalografica{ficha.pdf}
\imprimirfolhadecertificacao
 % first page, BU and others stuffs
\begin{dedicatoria}
  Este trabalho é dedicado à wikipedia e ao stackoverflow. 
\end{dedicatoria}

\begin{agradecimentos}
  A todos e a todas que me ajudaram.
\end{agradecimentos}

\begin{epigrafe}
  Uma frase linda\\
\end{epigrafe} % dedication, thanks  epigraph
\begin{resumo}[Resumo]
  Aqui deve ser inserido um resumo de 150 a 500 palavras (limitação de tamanho dada pela BU). A linguagem deve ser português e a hifenização já foi alterada. O resumo em português deve preceder o resumo em inglês, mesmo que o trabalho seja escrito em inglês. A BU também diz que deve ser usada a voz ativa e o discurso deve ser na 3ª pessoa. A estrutura do resumo pode ser similar a estrutura usada em artigos: Contexto -- Problema -- Estado da arte -- Solução proposta  -- Resultados.

  % Atenção! a BU exige separação através de ponto (.). Ela recomanda de 3 a 5 keywords
  \vspace{\baselineskip} 
  \textbf{Palavras-chave:} Palavra-chave. Ponto como separador. Bla.
\end{resumo}


\begin{resumo}[Resumo Estendido]
  \section*{Introdução} 
  A hifenização é alterada para \texttt{brazil}, mesmo para documentos em inglês. Descrever brevemente esses itens exigidos pela BU. Como a RN 95/CUn/2017 é mais recente e impõe outras regras a revelia de regimentos e regulamentos, é mais sábio obedecê-la. Lembre que esse resumo estendido deve term entre 2 e 5 páginas.
  
  \lipsum[1]
  \section*{Objetivos} 
  \lipsum[2]
  \section*{Metodologia} 
  \lipsum[3]
  \section*{Resultados e Discussão} 
  \lipsum[4]
  \section*{Considerações Finais} 
  \lipsum[5]

  \vspace{\baselineskip}  % Atenção! manter igual ao resumo
  \textbf{Palavras-chave:} Palavra-chave. Outra Palavra-chave composta. Bla.
\end{resumo}

\begin{abstract}
In the last few years, post-quantum cryptography has received much attention. NIST is running a competition to select some post-quantum schemes as standard. As a consequence, implementations of post-quantum schemes have become important and with them side-channel attacks. In this paper, we show a timing attack on a code-based scheme which was submitted to the NIST competition. This timing attack recovers secret information because of a timing variance in finding roots in a polynomial. We present five algorithms to find roots that are protected against timing exploitation.\vspace{\baselineskip} 
  
\textbf{Keywords:} {Side-channel Attack. Post-quantum Cryptography. Code-based Cryptography. Roots Finding.}
\end{abstract} % resumo e abstract
\listoffigures*

\begin{listadesimbolos}
$\dots$ & pontos \\
oi & techau
\end{listadesimbolos}

\tableofcontents*% % lists and table of contents


%%%%%%%%%%%%%%%%%%%%%%%%%%%%%%%%%%%%%%%%%%%%%%%%%%%%%%%%%%%%%%%%%%%%
%%% Corpo do texto                                               %%%
%%%%%%%%%%%%%%%%%%%%%%%%%%%%%%%%%%%%%%%%%%%%%%%%%%%%%%%%%%%%%%%%%%%%
\textual%

\chapter{Introduction}
\label{ch:intro}
The \cite{mceliece1978public}
\section{Objectives}
\subsection{General objectives}
\subsection{Specific objectives}
\section{Methodology}
\section{Scientific contribution}
\section{Organization}


\chapter{Mathematical Background}
\label{ch:math}
\section{Cryptography}
\section{Coding theory}
\section{Side-channel attacks}

\chapter{Code-based cryptography}
\label{ch:code-based}
\section{McEliece Cryptossystem}
\subsection{Definitions}
\subsection{Key Generation}
\subsection{Encryption}
\subsection{Decryption}
\section{Attack on BIGQUAKE}
\subsection{BIGQUAKE Submission}
\subsection{Timing Side-channel Attack on BIGQUAKE}

\chapter{Root finding methods}
\label{ch:roots}
\section{Exhaustive search}
\section{Berlekamp Trace Algorithm}
\section{Linearized Polynomials}
\section{Successive Resultant Algorithm}
\section{Cantor and Zazenhaus Algorithm}


\chapter{Analysis}
\label{ch:analysis}
\section{Performance Analysis}
\section{Time variance Analysis}
\section{Security Analysis}

\chapter{Final Considerations}
\label{ch:final}
The \cite{mceliece1978public}
\section{Objectives}
\subsection{General objectives}
\subsection{Specific objectives}
\section{Methodology}
\section{Scientific contribution}
\section{Organization}


%%%%%%%%%%%%%%%%%%%%%%%%%%%%%%%%%%%%%%%%%%%%%%%%%%%%%%%%%%%%%%%%%%%%
%%%                   Elementos pós-textuais                     %%%
%%%%%%%%%%%%%%%%%%%%%%%%%%%%%%%%%%%%%%%%%%%%%%%%%%%%%%%%%%%%%%%%%%%%
\postextual
\bibliography{main}
\end{document}
