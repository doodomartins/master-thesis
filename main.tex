\documentclass[brazil, english]{ufsc-thesis}

\usepackage[T1]{fontenc} % fontes
\usepackage[utf8]{inputenc} % UTF-8
\usepackage{lipsum} % Gerador de texto
\usepackage{pdfpages} % Inclui PDF externo (ficha catalográfica)
\usepackage{tikz} % tikz
\usepackage{amsfonts}
\usepackage{amsmath}
\usepackage[ruled,linesnumbered]{algorithm2e} % algorithms
\usepackage{tabularx} % protocol
\usepackage{hyperref}
\usepackage{listings}


%%%%%%%%%%%%%%%%%%%%%%%%%%%%%%%%%%%%%%%%%%%%%%%%%%%%%%%%%%%%%%%%%%%%
%%% Configurações da classe (dados do trabalho)                  %%%
%%%%%%%%%%%%%%%%%%%%%%%%%%%%%%%%%%%%%%%%%%%%%%%%%%%%%%%%%%%%%%%%%%%%
\titulo{Root finding techniques in binary finite fields}
\autor{Douglas Marcelino Beppler Martins}
\data{1 de Agosto de 2019}
\instituicao{Universidade Federal de Santa Catarina}
\centro{Centro Tecnológico}
\programa{Programa de Pós-Graduação em Ciência da Computação}
\dissertacao
\local{Florianópolis}
\titulode{Mestre em Ciência da Computação}

%%%%%%%%%%%%%%%%%%%%%%%%%%%%%%%%%%%%%%%%%%%%%%%%%%%%%%%%%%%%%%%%%%%%
%%% Membros da banca (Dr. depois)                                %%%
%%%%%%%%%%%%%%%%%%%%%%%%%%%%%%%%%%%%%%%%%%%%%%%%%%%%%%%%%%%%%%%%%%%%
\orientador{Prof. Ricardo Felipe Custódio, Dr.}
\membrobanca{Prof. Daniel Panario, Dr.}
{Carleton University}
\membrobanca{Prof. Jean Everson Martina, Dr.}
{Universidade Federal de Santa Catarina}
\membrobanca{Prof. Huifen Chan, Dr.}
{Universidade Federal de Santa Catarina}
\coordenadora{Prof. Vania Bogorny, Dr}

\DeclareMathAlphabet{\mathcal}{OMS}{cmsy}{m}{n}


%%%%%%%%%%%%%%%%%%%%%%%%%%%%%%%%%%%%%%%%%%%%%%%%%%%%%%%%%%%%%%%%%%%%
%%% Protocol environment                                         %%%
%%%%%%%%%%%%%%%%%%%%%%%%%%%%%%%%%%%%%%%%%%%%%%%%%%%%%%%%%%%%%%%%%%%%
\newcounter{protocol}
\newenvironment{protocol}[1]
  {\noindent
   \tabularx{\linewidth}{@{} X @{}}
    \hline
    \vspace{-3mm}
    \refstepcounter{protocol}\hspace{7.5mm}\textbf{Protocol \theprotocol:} #1 \\
    \hline}
    {\endtabularx}

\newcommand{\sbline}{\\[.5\normalbaselineskip]}% small blank line


\newtheorem{definition}{Definition}[chapter]
\newtheorem{lemma}{Lemma}[chapter]
\newtheorem{example}{Example}[chapter]


\begin{document}
%%%%%%%%%%%%%%%%%%%%%%%%%%%%%%%%%%%%%%%%%%%%%%%%%%%%%%%%%%%%%%%%%%%%
%%% Principais elementos pré-textuais                            %%%
%%%%%%%%%%%%%%%%%%%%%%%%%%%%%%%%%%%%%%%%%%%%%%%%%%%%%%%%%%%%%%%%%%%%
% Inicia parte pré-textual do documento capa, folha de rosto, folha de
% aprovação, aprovação, resumo, lista de tabelas, lista de figuras, etc.
\pretextual%
\imprimircapa%
\imprimirfolhaderosto*
%\protect\incluirfichacatalografica{tex/pre_tex/ficha.pdf}
\imprimirfolhadecertificacao
 % first page, BU and others stuffs
\begin{dedicatoria}
  To my grandfather, Marcelino Vieira Filho.
\end{dedicatoria}

\begin{agradecimentos}
    I wish to say tank you...
\end{agradecimentos}

\begin{epigrafe}
  Uma frase linda\\
\end{epigrafe} % dedication, thanks  epigraph
\begin{resumo}[Resumo]
  Aqui deve ser inserido um resumo de 150 a 500 palavras (limitação de tamanho dada pela BU). A linguagem deve ser português e a hifenização já foi alterada. O resumo em português deve preceder o resumo em inglês, mesmo que o trabalho seja escrito em inglês. A BU também diz que deve ser usada a voz ativa e o discurso deve ser na 3ª pessoa. A estrutura do resumo pode ser similar a estrutura usada em artigos: Contexto -- Problema -- Estado da arte -- Solução proposta  -- Resultados.

  % Atenção! a BU exige separação através de ponto (.). Ela recomanda de 3 a 5 keywords
  \vspace{\baselineskip} 
  \textbf{Palavras-chave:} Palavra-chave. Ponto como separador. Bla.
\end{resumo}


\begin{resumo}[Resumo Estendido]
  \section*{Introdução} 
  A hifenização é alterada para \texttt{brazil}, mesmo para documentos em inglês. Descrever brevemente esses itens exigidos pela BU. Como a RN 95/CUn/2017 é mais recente e impõe outras regras a revelia de regimentos e regulamentos, é mais sábio obedecê-la. Lembre que esse resumo estendido deve term entre 2 e 5 páginas.
  
  \lipsum[1]
  \section*{Objetivos} 
  \lipsum[2]
  \section*{Metodologia} 
  \lipsum[3]
  \section*{Resultados e Discussão} 
  \lipsum[4]
  \section*{Considerações Finais} 
  \lipsum[5]

  \vspace{\baselineskip}  % Atenção! manter igual ao resumo
  \textbf{Palavras-chave:} Palavra-chave. Outra Palavra-chave composta. Bla.
\end{resumo}

\begin{abstract}
  Enlish version of the plain ``resumo'' above. Done with environment
  \texttt{abstract}. Hyphenization is automatically changed to english.

  \vspace{\baselineskip} 
  \textbf{Keywords:} Keyword. Another Compound Keyword. Bla.
\end{abstract} % resumo e abstract
\listoffigures*

\begin{listadesimbolos}
$\dots$ & pontos \\
oi & techau
\end{listadesimbolos}

\tableofcontents*% % lists and table of contents


%%%%%%%%%%%%%%%%%%%%%%%%%%%%%%%%%%%%%%%%%%%%%%%%%%%%%%%%%%%%%%%%%%%%
%%% Corpo do texto                                               %%%
%%%%%%%%%%%%%%%%%%%%%%%%%%%%%%%%%%%%%%%%%%%%%%%%%%%%%%%%%%%%%%%%%%%%
\textual

\chapter{Introduction}
\label{ch:intro}
Communications thought electronic devices require privacy. This privacy between two parts is made with key agreements and key encapsulation protocols or public-key algorithms. During a several years, this protocols was designed over the classical cryptography, which was based on number theory problems. Nowadays, the integer factorization and the discrete logarithm are consider secure. However, the quantum algorithm proposed by Shor~\cite{shor1999polynomial} provides an polynomial time algorithm to solve this numerical problems in a quantum computer. Besides, the recent and fast advances on quantum computing makes necessary the study of new cryptography primitives. 

Furthermore, in recent years, the area of post-quantum cryptography has received considerable attention, especially because of the call by the National Institute of Standards and Technology (NIST) for  standardization of post-quantum schemes. On this call, NIST did not give restrictions about specific hard problems. However, most schemes for the Key Encapsulation Mechanism (KEM) are lattice- and code-based. The latter type is centered around coding theory and includes one of the oldest unbroken cryptosystems, due to McEliece~\cite{mceliece1978public}.

This classical algorithm uses an error-correcting code that able to recovery errors added to a message. This is made through the redundancy added to the original message. Using the idea behind the coding theory, and protecting some parts of the code, only who has knowledge of the code was able to recover the original message. Therefore we can construct schemes based on coding theory which are safe against quantum and classical attacks. Their implementation could not be secure. One of the requirements for those proposals is that they are resistant to all known cryptanalysis methods. In particular, cryptosystems need to avoid side-channel attacks.

There are different ways to apply side-channel attacks to a cryptosystem. As an example, an attacker can measure the execution time of the operations performed by an algorithm and, based on these measurements, estimate some secret information of the scheme. Although the attack scenario it is non-trivial, side-channels attacks are a dangerous mechanism that a cryptosystem needs to taken care of this attack.
 
In code-based cryptography, timing attacks on the decryption process are essentially done during the retrieval of the Error Locator Polynomial (ELP), as is shown by~\cite{shoufan2009timing}. The attack is usually done in the process of evaluating the polynomials, performed to identify the roots. This attack was demonstrated first in~\cite{shoufan2009timing} and later in an improved version in~\cite{bucerzan2017improved}.

In~\cite{strenzke2012fast} the authors made a survey of algorithms and compared they performances to find roots efficiently in code-based cryptosystems. However, the author shows only timings in different types of implementations, and selects the one which has the least timing variability. In other words, the author does not present an algorithm to find the roots in constant time and therefore eliminate the attack, as remarked in~\cite{strenzke2013efficiency}.

The algorithms presented in~\cite{strenzke2012fast} were not created with constant behavior of the implementations in mind. The authors present two optimizations in the exhaustive search method, but these do not affect time variations while the algorithms are in execution. Strenzke further proposes other improvements, some of them in the classical Berlekamp Trace Algorithm~\cite{berlekamp1970factoring} and also in the algorithm proposed in~\cite{fedorenko2002finding}. However, none of these implementations focus on constant-time behaviors or protect the implementation, and thus may leak sensitive information in the decoding process of the McEliece cryptosystem.

The root-finding implementations presented in~\cite{chou2017mcbits, bernstein2013mcbits} use Fast Fourier Transforms (FFT) and, while efficient, they are built and optimized for $\mathbb{F}_{2^{13}}$. We propose a more generic implementation that does not require specific optimizations on the underlying finite field arithmetic. Additionally, this approach takes advantage of a particular computer architecture and uses the fact that we can evaluate multiple points in parallel. We are also interested in approaches that could avoid side channel attacks in any architecture. 

In this work, to evidence the power of a timing side-channel attack, we present a implementation of the Strenzke attack over a NIST Round 1 submission. And the main focus of this works was propose strategies to make the execution time of the aforementioned algorithms constant, additionally, we propose the use of probabilistic algorithms to achieve a safe root computation. One of the strategies is to write the algorithms in an iterative way, eliminating all recursions. We also use permutations and simulated operations to uncouple possible measurements of side effects of the data being measured. Also, 

\section{Objectives}
The main goal of this work is to find alternatives to perform the decoding process of McEliece Cryptosystem in a safe way avoiding timing side-channel attacks. To achieve this, we are interested in building a constant way to compute roots of error locator polynomial, or remove the relation between the polynomial to be factorized and the execution time of the algorithm.


\subsection{Specific objectives}
\begin{enumerate}[label=\roman*., itemsep=1pt]
    \item Perform an timing side-channel attack against a code-based cryptosystem which has non-constant root extraction;
    \item Selection of main methods applied to compute the roots of a polynomial in code-based cryptosystems;
    \item Time variation measurement of works selected in the previous item;
    \item Propose strategies to achieve a secure way to compute roots;
    \item Evaluation of the new time variations of the algorithms;
\end{enumerate}


\section{Methodology}
\section{Scientific contribution}
The timing side-channel attack performed in Chapter~\ref{ch:code-based} and the countermeasures proposed in Chapter~\ref{ch:roots} results in the following paper:

\begin{itemize}
    \item  MARTINS, D.; BANEGAS, G.; CUSTÓDIO, R. Don’t Forget Your Roots: Constant-Time Root Finding over $\mathbb{F}_2^m$. In: International Conference on Cryptology and Information Security in Latin America.  2019. p. 109-129. Available in: \url{https://doi.org/10.1007/978-3-030-30530-7\_6}
\end{itemize}
\section{Organization}



% % 



% %%%%%%%%%%%%%%%%%%%%%%%%%%%%%%%%%%%%%%%%%%%%% %








% Communications thought electronic devices require privacy. This privacy between two parts is made with key agreements and key encapsulation protocols or public-key algorithms. 

% The context of this work relies on the fact that  Shor's algorithm opens a lack of security in all current cryptosystem. However, there exists a few class of the algorithm that is still secure, against a classical and a quantum computer. One of the most relevant classes of algorithm, called code-based algorithms, are based on the classical work proposed by Robert J. McEliece \cite{mceliece1978public}. 

% This classical algorithm uses an error-correcting code that able to recovery errors added to a message. This is made through the redundancy added to the original message. Using the idea behind the coding theory, and protecting some parts of the code, only who has knowledge of the code was able to recover the original message, we can construct a scheme that is secure in a quantum era. Therefore we can construct schemes based on code secure. Their implementation could not be secure. Based on this fact, many cryptosystems are creating ways to blinding their implementations. 

% In case of schemes based on errors correcting codes, more specific on schemes that uses algebraic decoding, we has some time leakage on the computation of the roots. The main task was, how to compute the roots of an polynomial over finite fields without leak time information. 

% We study, propose modifications, compare and analyze five algorithms that compute the roots of a polynomial.

\chapter{Scientific method}
\label{ch:method}
\input{tex/2-method}

\chapter{Mathematical Background}
\label{ch:math}
This chapter explain the necessary background to understand this works. A brief mathematical and coding theory concepts, followed by a explanation of cryptography primitives used in this works and finally, a explanation of side-channel attacks.
\section{Algebraic structures}
\subsection{Finite fields}
Explain main proprieties, main operations
\subsection{Finite field extensions}

\subsection{Polynomials over finite fields}
\section{Coding theory}
\subsection{Linear codes}
\subsection{Goppa codes}
\section{Cryptography primitives}
\subsection{Public-key cryptography}
\subsection{Key encapsulation mechanisms}
\section{Side-channel attacks}
\subsection{Timing side-channel attacks}


\chapter{Code-based cryptography}
\label{ch:code-based}
The rising of quantum computing in the last few year became the attention of researchers to schemes that uses cryptography primitives different from discrete logarithm and number factorization. As mentioned before in Chapter \ref{ch:intro}, the code-based cryptosystems are thoose who 


\section{McEliece Cryptosystem}
\subsection{Definitions}
\subsection{Decryption}
\section{BIGQUAKE}
\subsection{Submission overview}
\subsection{Timing side-channel attack}

\chapter{Root finding techniques and countermeasures}
\label{ch:roots}
As argued, the leading cause of information leakage in the decoding algorithm is the process of finding the roots of the ELP. In general, the time needed to find these roots varies, often depending on the roots themselves. Thus, an attacker who has access to the decoding time can infer these roots, and hence get the vector of errors $e$. One example of this attack are presented in Section~\ref{sec:attack}, where we perform an side-channel attack over the decryption part of a cryptosystem and we recover the secret exchange between two parts of the protocol.

Factoring a polynomial it is a well studied problem on finite fields area. More recently, with the rising of the code-based area, some works starts to deal with the factorization problem on the decoding process in order to avoid timing side-channel attacks~\cite{shoufan2009timing, bucerzan2017improved}. In our systematic review, presented on Chapter~\ref{ch:method}, we select four methods used on code-based cryptosystem to analyse and to propose modifications in theirs implementation to made then available on cryptosystems applications.

The adjustments were made in the following algorithms to find roots: exhaustive search, linearized polynomials, Berlekamp trace algorithm (BTA), and successive resultant algorithm (SRA). Also, we present the Cantor-Zassenhaus method, as it has a probabilistic algorithm that made use of an random selection of polynomials in their execution, we made an security analysis over this algorithm to evaluate their blindness on a side-channel attack scenario. The description of each algorithm are presented in the next sections and the analysis are presented on Chapter~\ref{ch:comparison}.

\section{Exhaustive search}
The exhaustive search, also called Chien Search~\cite{chien1964cyclic}, is a direct method, in which the evaluation of $f$ for all the elements in $\mathbb{F}_{2^m}$ is performed. A root is found whenever the evaluation result is zero. This method is acceptable for small fields and can be made efficient with a parallel implementation. The greatest drawback of this method it was the complexity of this algorithm, and this is the reason because their are not widely used. However, we consider that their are still relevant, it is because some NIST proposals are based on a small finite field, enabling their usage. 

Algorithm~\ref{alg:exhaustive} describes the exhaustive search. For every element in $\mathbb{F}_{2^m}$ we evaluate the polynomial and check if it was a root or not, and this method leaks information when we found a root. This is because whenever we found, i.e., $dummy = 0$, an extra operation is performed by adding the root founded on the returning list $R$. In this way, the attacker can infer from this additional time that a root was found, thus providing ways to obtain data that should be secret.

\begin{figure}[ht]
\begin{algorithm}[H]
 \KwData{$p(x)$ as univariate polynomial over $\mathbb{F}_{2^m}$ with $d$ roots, $A = [a_0, \ldots, a_{n-1}]$ as all elements in $\mathbb{F}_{2^m}$, $n$ as the length of $A$.}
 \KwResult{$R$ as a set of roots of $p(x)$.}
 $R \gets \emptyset$\;
\For{$i\gets0$ \KwTo $n-1$}{
    $dummy \gets p(A[i])$\;
   \If{$dummy == 0$}{
        $R.add(A[i])$\;
    }
}
\Return $R$\;
  \caption{Exhaustive search algorithm for finding roots of a univariate polynomial over $\mathbb{F}_{2^m}$.}\label{alg:exhaustive}
\end{algorithm}
\end{figure}

One solution to avoid this leakage is to permute the elements of vector $A$ before start the evaluation. Using this technique, an attacker can identify the extra operation, but without learning any secret information. In our case, we use the Fisher-Yates shuffle~\cite{black2005fisher} for shuffling the elements of vector $A$. In~\cite{wang2018fpga}, the authors show an implementation of the shuffling algorithm safe against timing attacks, thus we can build and exhaustive search without leak information against a side-channel attack. Algorithm~\ref{alg:exhaustive_permuted} shows the permutation of the elements and the computation of the roots.

\begin{figure}[ht]
\begin{algorithm}[H]
 \KwData{$p(x)$ as a univariate polynomial over $\mathbb{F}_{2^m}$ with $d$ roots, $A = [a_0, \ldots, a_{n-1}]$ as all elements in $\mathbb{F}_{2^m}$, $n$ as the length of $A$.}
 \KwResult{$R$ as a set of roots of $p(x)$.}
  permute$(A)$\;
 $R \gets \emptyset$\;
\For{$i\gets0$ \KwTo $n-1$}{
    $dummy \gets p(A[i])$\;
   \If{$dummy == 0$}{
        $R.add(A[i])$\;
    }
}
\Return $R$\;
 \caption{Exhaustive search algorithm with a countermeasure for finding roots of an univariate polynomial over $\mathbb{F}_{2^m}$.}
  \label{alg:exhaustive_permuted}
\end{algorithm}
\end{figure}

Using this approach, we add one extra step to the algorithm. However, this permutation blurs the sensitive information of the algorithm, making the usage of Algorithm~\ref{alg:exhaustive_permuted} slightly harder for the attacker to acquire timing leakage. For this case, we want to avoid the addition of an \texttt{else} clause adding the elements that are not a root in other list. This addition will absolutely reduce the time variance on the execution of the algorithm, but it could open a lack for cache attacks~\cite{dan2005}.


\section{Berlekamp Trace Algorithm}
Our second strategy was the classical Berlekamp factoring algorithm~\cite{berlekamp1970factoring}. Berlekamp presents an efficient algorithm to factor a polynomial, which can be used to find its roots. We call this algorithm \emph{Berlekamp trace algorithm} since it works with a 

The trace function defined as $Tr(x) = x + x^{2} + x^{2^{2}} + \dots + x^{2^{m-1}}$. It is possible to change BTA for finding roots of a polynomial $p(x)$ using $\beta = \{\beta_1, \beta_2, \ldots, \beta_m\}$ as a standard basis of $\mathbb{F}_{2^{m}}$, and then computing the greatest common divisor between $p(x)$ and $Tr(\beta_0 \cdot x)$. After that, it starts a recursion where BTA performs two recursive calls; one with the result of gcd algorithm and the other with the remainder of the division $p(x) / \gcd(p(x), Tr(\beta_i \cdot x))$. 

The base case is when the degree of the input polynomial is smaller than one. In this case, BTA returns the root, by getting the independent term of the polynomial. In summary, the BTA is a divide and conquer like algorithm since it splits the task of computing the roots of a polynomial $p(x)$ into the roots of two smalls polynomials. All this steps are describe in Algorithm~\ref{alg:bta}.


\begin{figure}[ht]
\begin{algorithm}[H]
 \KwData{$p(x)$ as a univariate polynomial over $\mathbb{F}_{2^m}$ and i.}
 \KwResult{The set of roots of $p(x)$.}
    \If{$deg(p(x)) \leq 1$}{
        \Return root of $p(x)$\;
    }
    $p_{0}(x) \gets gcd(p(x), Tr(\beta_{i}\cdot x))$\;
    $p_{1}(x) \gets p(x) / p_{0}(x)$ \;
\Return $BTA(p_{0}(x), i + 1) \cup BTA(p_{1}(x), i + 1)$\;
 \caption{Berlekamp Trace Algorithm -- $BTA(p(x), i)-rf$.}
  \label{alg:bta}
\end{algorithm}
\end{figure}

As we can see, a direct implementation of Algorithm~\ref{alg:bta} has no constant execution time. The recursive behavior may leak information about the characteristics of roots in a side-channel attack. Additionally, in our experiments, we noted that the behavior of the gcd with the trace function may result in a polynomial with the same degree. Therefore, BTA will divide this input polynomial in a future call with a different basis. Consequently, there is no guarantee of a constant number of executions. 

In order to avoid the nonconstant number of executions, here referred as $BTA-it$, we propose an iterative implementation of Algorithm~\ref{alg:bta}. In this way, our proposal iterates in a fixed number of iterations instead of calling itself until the base case. The main idea is not changed; we still divide the task of computing the roots of a polynomial $p(x)$ into two smaller instances. However, we change the approach of the division of the polynomial. Since we want to compute the same number of operations independent of the degree of the polynomial, we perform the gcd with a trace function for all basis in $\beta$, and choose a division that results in two new polynomials with approximate degree.

This new approach allows us to define a fixed number of iterations for our version of BTA. Since we always divide into two small instances, we need $t-1$ iterations to split a polynomial of degree $t$ in $t$ polynomials of degree $1$. Algorithm~\ref{alg:ibta} presents this approach.


\begin{algorithm}[ht]
 \KwData{$p(x)$ as an univariate polynomial over $\mathbb{F}_{2^m}$, $t$ as number of expected roots.}
 \KwResult{The set of roots of $p(x)$.}
    $g \gets \{p(x)\};$ \tcp{The set of polynomials to be computed}
    \For{$k \gets 0$ \KwTo $t$}{
        $current = g.pop()$\;
        Compute $candidates$ $=$ $gcd(current, Tr(\beta_{i}\cdot x))$ $\forall$ $\beta_{i}$ $\in$ $\beta$\;
        Select $p_{0}$ $\in$ $candidates$ such as $p_{0}.degree$ $\simeq$ $\frac{current}{2}$\;
        $p_{1}(x) \gets current / p_{0}(x)$ \;
        \If{$p_{0}.degree == 1$}{
            $R.add($root of $p_{0})$
        } \Else{
            $g.add(p_{0})$\;
        }
        \If{$p_{1}.degree == 1$}{
            $R.add($root of $p_{1})$
        } \Else{
            $g.add(p_{1})$\;
        }
    }
    \Return{$R\;$}
 \caption{Iterative Berlekamp Trace Algorithm -- $BTA(p(x))-it$.}
  \label{alg:ibta}
\end{algorithm}

Algorithm~\ref{alg:ibta} extracts a root of the polynomial when the variable $current$ has a polynomial with degree equal to one. If this degree is greater than one, then the algorithm needs to continue dividing the polynomial until it finds a root. The algorithm does that by adding the polynomial in a stack and reusing this polynomial in a division. 

\section{Linearized Polynomials}
The second countermeasure proposed is based on linearized polynomials. The authors in \cite{fedorenko2002finding} propose a method to compute the roots of a polynomial over $\mathbb{F}_{2^m}$, using a particular class of polynomials, called linearized polynomials. In \cite{strenzke2012fast}, this approach is a recursive algorithm which the author calls ``dcmp-rf''. In our solution, however, we present an iterative algorithm. We define linearized polynomials as follows:

\begin{definition}
A polynomial $\ell(y)$ over $\mathbb{F}_{2^m}$ is called a linearized polynomial if
\begin{equation}
    \ell(y) = \sum_i c_iy^{2^i},
\end{equation}
where $c_i \in \mathbb{F}_{2^m}$.
\end{definition}
In addition, from~\cite{truong2001fast}, we have Lemma~\ref{lemma:lin} that describes the main property of linearized polynomials for finding roots.
\begin{lemma}
\label{lemma:lin}
    Let $y \in \mathbb{F}_{2^m}$ and let $\alpha^0, \alpha^1, \ldots, \alpha^{m-1}$ be a standard basis over $\mathbb{F}_2$. If
    \begin{equation}
        y = \sum_{k=0}^{m-1} y_k\alpha^k, y_k \in \mathbb{F}_2
    \end{equation}
    and $\ell(y) =\sum_j c_jy^{2^j}$, then
      \begin{equation}
        \ell(y) = \sum_{k=0}^{m-1} y_k\ell(\alpha^k).
    \end{equation}
\end{lemma}

We call $A(y)$ over $\mathbb{F}_{2^m}$ as an affine polynomial if $A(y) = \ell(y) + \beta$ for $\beta \in \mathbb{F}_{2^m}$, where $\ell(y)$ is a linearized polynomial.

We can illustrate a toy example to understand the idea behind finding roots using linearized polynomials.
\begin{example}\label{ex:1}
Let us consider the polynomial $f(y) = y^2 + (\alpha^2+1)y + (\alpha^2 +\alpha +1)y^0$ over $\mathbb{F}_{2^3}$ and $\alpha$ are elements in $\mathbb{F}_2[x]/ x^3+x^2+1$. Since we are trying to find roots, we can write $f(y)$ as
 $$ y^2 + (\alpha^2+1)y + (\alpha^2 +\alpha +1)y^0 = 0$$
 or
\begin{equation}\label{eq:example_1}
    y^2 + (\alpha^2+1)y   = (\alpha^2 +\alpha +1)y^0
\end{equation}
We can point that on the left hand side of Equation~\ref{eq:example_1}, $\ell(y) = y^2 + (\alpha^2+1)y$ is a linearized polynomial over $\mathbb{F}_{2^3}$ and Equation~\ref{eq:example_1} can be expressed just as
\begin{equation}\label{eq:example_1_2}
    \ell(y) = \alpha^2 +\alpha +1
\end{equation}
If $y = y_2\alpha^2 + y_1\alpha + y_0 \in \mathbb{F}_{2^3}$ then, according to Lemma~\ref{lemma:lin}, Equation~\ref{eq:example_1_2} becomes
\begin{equation}\label{eq:example_1_3}
    y_2\ell(\alpha^2) + y_1\ell(\alpha) + y_0\ell(\alpha^0) = \alpha^2 +\alpha +1
\end{equation}
We can compute $\ell(\alpha^0),\ell(\alpha)$ and $\ell(\alpha^2)$ using the left hand side of Equation~\ref{eq:example_1} and we have the following values
\begin{equation}\label{eq:example_1_4}
    \begin{split}
        \ell(\alpha^0) & = (\alpha^0)^2 + (\alpha^2+1)(\alpha^0) = \alpha^2+1 + 1 = \alpha^2 \\
        \ell(\alpha) & = (\alpha)^2 + (\alpha^2+1)(\alpha) = \alpha^2 + \alpha^2+ \alpha + 1 = \alpha + 1 \\
        \ell(\alpha^2) & = (\alpha^2)^2 + (\alpha^2+1)(\alpha^2) = \alpha^4 +\alpha^4 +  \alpha^2 = \alpha^2.
    \end{split}
\end{equation}
A substitution of Equation~\ref{eq:example_1_4} into Equation~\ref{eq:example_1_3} gives us
\begin{equation}\label{eq:example_1_5}
     (y_2+y_0)\alpha^2 + (y_1)\alpha + (y_1)\alpha^0 = \alpha^2 +\alpha +1
\end{equation}
Equation~\ref{eq:example_1_5} can be expressed as a matrix in the form
\begin{equation}\label{eq:example_1_6}
    \begin{bmatrix} y_2 & y_1 & y_0 \end{bmatrix}
    \begin{bmatrix}
    1 & 0 & 0 \\
    0 & 1 & 1 \\
    1 & 0 & 0
    \end{bmatrix}
    =
    \begin{bmatrix} 1 & 1 & 1 \end{bmatrix}.
\end{equation}
If one solves simultaneously the linear system in Equation~\ref{eq:example_1_6} then the results are the roots of the polynomial given in Equation~\ref{eq:example_1}. From Equation~\ref{eq:example_1_5}, one observes that the solutions are $y=110$ and $y=011$, which can be translated to $\alpha + 1$ and $\alpha^2 + \alpha$.
\end{example}


Fortunately, the authors in \cite{fedorenko2002finding} provide a generic decomposition for finding affine polynomials. In their work, each polynomial in the form $F(y) = \sum_{j=0}^{t} f_jy^j$ for $f_j \in \mathbb{F}_{2^m}$ can be represented as
\begin{equation}
\label{eq:f_y}
    F(y) = f_3y^3 + \sum_{i=0}^{\lceil (t-4)/5 \rceil} y^{5i}(f_{5i} + \sum_{j=0}^{3} f_{5i+2^j}y^{2^j})
\end{equation}
After that, we can summarize all the steps as Algorithm~\ref{alg:linearized}. The function ``generate($m$)'' refers to the generation of the elements in $\mathbb{F}_{2^m}$ using Gray codes, see \cite{savage1997survey} for more details about Gray codes. Algorithm~\ref{alg:linearized} presents a countermeasure in the last steps of the algorithm, i.e., we added a dummy operation for blinding if $X[j]$ is a root of polynomial $F(x)$.

\begin{figure}
\begin{algorithm}[H]
 \KwData{$F(x)$ as a univariate polynomial over $\mathbb{F}_{2^m}$ with degree $t$ and $m$ as the extension field degree.}
 \KwResult{$R$ as a set of roots of $p(x)$.}
 $\ell^k_i \gets \emptyset$; $\ell_{is} \gets \emptyset$; $A^j_k \gets \emptyset$; $R \gets \emptyset$; $dummy \gets \emptyset$\;
 \If{$f_0  == 0$}{
 $R.append(0)$\;
 }
 \For{$i\leftarrow 0$ \KwTo $\lceil (t-4)/5 \rceil$}
 {
    $\ell_i(x) \gets 0$\;
    \For{$j\gets 0$ \KwTo $3$}{
      $\ell_i(x) \gets \ell_i(x) + f_{5i+2^j}x^{2^j}$\;
      }
    $\ell_{is}[i] \gets \ell_i(x)$\;
 }
 \For{$k\gets 0$ \KwTo $m-1$}{
    \For{$i\leftarrow 0$ \KwTo $\lceil (t-4)/5 \rceil$}
    {
        $\ell^k_i \gets \ell_{is}(\alpha^k)$\;
    }

 }
 $A^0_i \gets \emptyset$\;
 \For{$i\gets 0$ \KwTo $\lceil (t-4)/5 \rceil$}{
  $A^0_i \gets f_{5i}$\;
 }

 $X \gets \text{generate}(m)$\;
 \For{$j\gets 1$ \KwTo $2^m - 1$}{
    \For{$i\gets 0$ \KwTo $\lceil (t-4)/5 \rceil$}{
        $A \gets A^{j-1}_i$\;
        $A \gets A + \ell^{\delta(X[j], X[j-1])}_i$\;
        $A^j_i \gets A$\;
    }
 }
\For{$j\gets 1$ \KwTo $2^m - 1$}{
    $result \gets 0$\;
    \For{$i\gets 0$ \KwTo $\lceil (t-4)/5 \rceil$}{
        $result = result + (X[j])^{5i}A^j_i$\;
    }
    $eval = result + f_3(X[j])^{3}$\;
    \eIf{$eval == 0$}{
        $R.append(X[j])$\;
    }{
        $dummy.append(X[j])$\;
    }
}
\Return $R$\;
 \caption{Linearized polynomials for finding roots over $\mathbb{F}_{2^m}$.}
  \label{alg:linearized}
\end{algorithm}
\end{figure}

\section{Successive Resultant Algorithm}
In \cite{petit2014finding}, the authors present an alternative method for finding roots in $\mathbb{F}_{p^m}$. Later on, the authors better explain the method in~\cite{petit2016finding}. The Successive Resultant Algorithm (SRA) relies on the fact that it is possible to find roots exploiting properties of an ordered set of rational mappings.

Given a polynomial $f$ of degree $d$ and a sequence of rational maps $K_1,\ldots, K_t$, the algorithm computes finite sequences of length $j \leq t+1$ obtained by successively transforming the roots of $f$ by applying the rational maps. The algorithm is as follows: Let $\{v_1,\ldots,v_m\}$ be an arbitrary basis of $\mathbb{F}_{p^m}$ over $\mathbb{F}_p$, then it is possible to define $m+1$ functions $\ell_0, \ell_1,\ldots, \ell_m$ from $\mathbb{F}_{p^m}$ to $\mathbb{F}_{p^m}$ such that
$$
\left \{
\begin{array}{l}
     \ell_0(z) = z\\
     \ell_1(z) = \prod_{i\in \mathbb{F}_p}\ell_0(z-iv_1)\\
     \ell_2(z) = \prod_{i\in \mathbb{F}_p}\ell_1(z-iv_2)\\
     \cdots \\
     \ell_m(z) = \prod_{i\in \mathbb{F}_p}\ell_{m-1}(z-iv_m)\\
\end{array}
\right.
$$
The functions $\ell_j$ are examples of linearized polynomials, as previously defined in Chapter~\ref{ch:math}. Our next step is to present the theorems from \cite{petit2014finding}. Check original work for the proofs.
\begin{theorem}\label{lemma_1}
\begin{itemize}
    \item[a)] Each polynomial $\ell_i$ is split and its roots are all elements of the vector space generated by $\{v_1, \ldots,v_i\}$. In particular, we have $\ell_n(z) = z^{p^m} -z$.
    \item[b)] We have $\ell_i(z)  = \ell_{i-1}(z)^p - a_i\ell_{i-1}(z)$ where $a := (\ell_{i-1}(v_i))^{p-1}$.
    \item[c)] If we identify $\mathbb{F}_{p^m}$ with the vector space $(\mathbb{F}_p)^m$, then each $\ell_i$ is a $p$-to-$1$ linear map of $\ell_{i-1}(z)$ and a $p^i$ to $1$ linear map of $z$.
\end{itemize}
\end{theorem}

From Theorem~\ref{lemma_1} and its properties, we can reach the following polynomial system:
\begin{equation}\label{eq:system_1}
    \left \{
\begin{array}{l}
    f(x_1) = 0\\
     x_j^p = a_jx_j = x_{j+1} \quad j=1,\ldots, m-1\\
     x_n^p - a_nx_n = 0
\end{array}
\right.
\end{equation}
where the $a_i \in \mathbb{F}_{p^n}$ are defined as in Theorem~\ref{lemma_1}. Any solution of this system provides us with a root of $f$ by the first equation, and the $n$ last equations together imply this root belongs to $\mathbb{F}_{p^n}$. From this system of equations,~\cite{petit2014finding} derives Theorem~\ref{lemma_2}.

\begin{theorem}\label{lemma_2}
Let $(x_1,x_2,\ldots,x_m)$ be a solution of the equations in Equation~\ref{eq:system_1}. Then $x_1 \in \mathbb{F}_{p^m}$ is a solution of $f$. Conversely, given a solution $x_1 \in \mathbb{F}_{p^m}$ of $f$, we can reconstruct a solution of all equations in Equation~\ref{eq:system_1} by setting $x_2 =x_1^p - a_1x_1$, etc.
\end{theorem}

In \cite{petit2014finding}, the authors present an algorithm for solving the system in Equation~\ref{eq:system_1} using resultants. The solutions of the system are the roots of polynomial $f(x)$. It is worth remarking that this algorithm is almost constant-time and hence we just need to protect the branches presented on it.


\section{Cantor-Zassenhaus Algorithm}


\chapter{Comparison}
\label{ch:comparison}
In this chapter we will present an analysis over the five presented method on previous section. The first two analysis has focus on the complexity and performance of the algorithms. However, we are not interested only in efficient methods, our main goal was achieve an method with no information leakage against a timing side-channel attack. Hence, we demonstrate an time-variance analysis for each proposed root-finding method. After that, we present an security analysis over the 

\section{Complexity analysis}
\section{Performance analysis}
\section{Time variance analysis}
\section{Security analysis}

\chapter{Final Considerations}
\label{ch:final}
In this thesis we propose countermeasures to be applied on root finding algorithms in order to achieve a decoding process without leak any sensitive information against a timing side-channel attack. We propose five methods, with different characteristics which can be applied to the root extraction task. Our proposals are based on reduce the time variance, by applying implementation techniques which aims to construct a algorithm without branches and with a constant behaviour. 

Before present the countermeasures, we present a 

\section{Future works}
For future work, we 


%%%%%%%%%%%%%%%%%%%%%%%%%%%%%%%%%%%%%%%%%%%%%%%%%%%%%%%%%%%%%%%%%%%%
%%%                   Elementos pós-textuais                     %%%
%%%%%%%%%%%%%%%%%%%%%%%%%%%%%%%%%%%%%%%%%%%%%%%%%%%%%%%%%%%%%%%%%%%%
\postextual
\bibliography{main}

\appendix
\chapter{Implementation Code}
\begin{lstlisting}[caption={Multiplication of two elements in $\mathbb{F}_{2^{12}}$ and inversion of an element in $\mathbb{F}_{2^{12}}$},label={lst:label},language=C]

#include <stdint.h>
typedef uint16_t gf;

//Multiplication between in0 and in1
gf gf_q_m_mult(gf in0, gf in1) {
    uint64_t i, tmp, t0 = in0, t1 = in1;
    //Multiplication
    tmp = t0 * (t1 & 1);
    for (i = 1; i < 12; i++)
        tmp ^= (t0 * (t1 & (1 << i)));
    //reduction
    tmp = tmp & 0x7FFFFF;
    //first step of reduction
    gf reduction = (tmp >> 12);
    tmp = tmp & 0xFFF;
    tmp = tmp ^ (reduction << 6);
    tmp = tmp ^ (reduction << 4);
    tmp = tmp ^ reduction << 1;
    tmp = tmp ^ reduction;
    //second step of reduction
    reduction = (tmp >> 12);
    tmp = tmp ^ (reduction << 6);
    tmp = tmp ^ (reduction << 4);
    tmp = tmp ^ reduction << 1;
    tmp = tmp ^ reduction;
    tmp = tmp & 0xFFF;
    return tmp;
}

// Multiplicative inverse of in
gf gf_inv(gf in) {
    gf tmp_11 = 0;
    gf tmp_1111 = 0;
    gf out = in;
    out = gf_sq(out); //a^2
    tmp_11 = gf_mult(out, in); //a^2*a = a^3
    out = gf_sq(tmp_11); //(a^3)^2 = a^6
    out = gf_sq(out); // (a^6)^2 = a^12
    tmp_1111 = gf_mult(out, tmp_11); //a^12*a^3 = a^15
    out = gf_sq(tmp_1111); //(a^15)^2 = a^30
    out = gf_sq(out); //(a^30)^2 = a^60
    out = gf_sq(out); //(a^60)^2 = a^120
    out = gf_sq(out); //(a^120)^2 = a^240
    out = gf_mult(out, tmp_1111); //a^240*a^15 = a^255
    out = gf_sq(out); // (a^255)^2 = 510
    out = gf_sq(out); //(a^510)^2 =  1020
    out = gf_mult(out, tmp_11); //a^1020*a^3 = 1023
    out = gf_sq(out); //(a^1023)^2 = 2046
    out = gf_mult(out, in); //a^2046*a = 2047
    out = gf_sq(out); //(a^2047)^2 = 4094
    return out;
}
\end{lstlisting}

\end{document}
