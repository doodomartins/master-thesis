Communications through electronic devices require privacy. This privacy between two parts is made with key agreements and key encapsulation protocols, or public-key algorithms. For several years, these protocols were designed over classical asymmetric cryptography, whose security is based on number theory problems. Nowadays, integer factorization and discrete logarithm, common examples of such problems, are considered secure. However, the quantum algorithm proposed by Shor~\cite{shor1999polynomial} are able to solve these numerical problems with a polynomial-time algorithm in a quantum computer. Besides, the recent and fast advances in quantum computing make necessary the study of new cryptography primitives, which are resistant to the Shor algorithm. 

In recent years, the area of post-quantum cryptography has received considerable attention, mainly because of the call by the National Institute of Standards and Technology (NIST) for the standardization of post-quantum cryptosystems. On this call, NIST did not give restrictions about specific hard problems. However, most of the submitted schemes for Key Encapsulation Mechanisms (KEM) are lattice- and code-based. The latter are centered around coding theory and includes one of the oldest unbroken cryptosystems, due to McEliece~\cite{mceliece1978public}.

This classical algorithm uses an error-correcting code that is able to recover errors added to a message. This process is achieved through the redundancy added to the original message. Using the idea behind coding theory, and protecting some parts of the code, only parties with knowledge of the code structure are able to recover the original message. Hence, we can construct schemes based on coding theory, which are safe against quantum and classical attacks. However, the scheme implementation may not be secure. One of the requirements for those proposals is that they are resistant to all known cryptanalysis methods. In particular, cryptosystems need to avoid side-channel attacks.

There are different ways to apply side-channel attacks to a code-based cryptosystem. As an example, an attacker can measure the execution time of the operations performed by an algorithm and, based on these time variation, estimate some secret information of the scheme. Although the attack scenario is non-trivial, side-channel attacks are a dangerous mechanism that a cryptosystem needs to consider this attack.
 
In code-based cryptography, timing attacks on the decryption process are mostly made during the retrieval of the Error Locator Polynomial (ELP). The attack is usually made in the process of evaluating the polynomials, performed to find its roots. This attack was demonstrated firstly in~\cite{shoufan2009timing} and later in an improved version in~\cite{bucerzan2017improved}.

In~\cite{strenzke2012fast}, the authors present a survey of algorithms and compared their performances to efficiently find roots in code-based cryptosystems. However, the author only shows timings spedup in different types of implementations. Additionally, they select the one which has the least timing variability. In other words, the author does not present an algorithm to find the roots in constant time and therefore eliminate the attack, as remarked in~\cite{strenzke2013efficiency}.

The algorithms presented in~\cite{strenzke2012fast} are not designed with constant behavior of the implementations in mind. The authors present two optimizations in the exhaustive search method, but these do not affect time variations while the algorithms are in execution. Strenzke further proposes other improvements, some of them in the classical Berlekamp Trace Algorithm~\cite{berlekamp1970factoring} and also in the algorithm proposed in~\cite{fedorenko2002finding}. However, none of these implementations focus on constant-time behaviors or protect the application, and thus may leak sensitive information in the decoding process of the McEliece cryptosystem.

The root-finding implementations presented in~\cite{chou2017mcbits, bernstein2013mcbits} use Fast Fourier Transforms (FFT) and, while efficient, they are built and optimized for $\mathbb{F}_{2^{13}}$. We propose a more generic implementation that does not require specific optimizations on the underlying finite field arithmetic. Additionally, this approach takes advantage of particular computer architecture and uses the fact that we can evaluate multiple points in parallel. We are also interested in approaches that could avoid side-channel attacks in any architecture. 

In this work, to evidence the endowment of a timing side-channel attack, we present an implementation of the Strenzke's attack over a NIST Round 1 submission, the BIGQUAKE. And the main focus of this works is to propose countermeasures to make the execution time of the aforementioned algorithms constant. Additionally, we propose the use of probabilistic algorithms to achieve a secure root computation. One of the countermeasures is to write the algorithms iteratively, eliminating all recursions. We also use permutations and simulated operations to uncouple possible measurements of the side effects of the data being measured.

\section{Objectives}
The main goal of this work is to find alternatives to perform the decoding process of McEliece Cryptosystem safely, avoiding timing side-channel attacks. To achieve this, we are interested in building a constant way to compute the roots of error locator polynomial or remove the relation between the polynomial to be factorized and the execution time of the algorithm.


\subsection{Specific objectives}
\begin{enumerate}[label=\roman*., itemsep=1pt]
    \item Perform a timing side-channel attack against a code-based cryptosystem which has non-constant root extraction;
    \item Selection of main methods applied to compute the roots of a polynomial in code-based cryptosystems;
    \item Time variation measurement of works selected in the previous item;
    \item Propose strategies to achieve a secure way to compute roots;
    \item Evaluation of the new time variations of the algorithms;
\end{enumerate}


\section{Methodology}
In order to accomplish the goals of this work, we present the methodology used in this work.
\begin{enumerate}
    \item Code-based cryptosystems overview: We first studied the theoretical background over the code-based cryptosystems. Starting from the first proposal by McEliece until the NIST standardization submissions;
    \item Literature review over timing side-channel attacks on code-based cryptosystems: The second step is to perform a literature review over side-channel attacks against code-based schemes. This literature review will be constructed through a search over the main journals and conferences;
    \item Perform a timing side-channel attack against a cryptosystem: to illustrate that a naive implementation is insecure against a timing side-channel attack, we will perform an attack against a code-based scheme;
    \item Literature review over root-finding methods: the main algorithms applied to compute the roots of a polynomial in code-based cryptosystems will be listed and individually studied; 
    \item Addition of different methods: Since the area of computing roots of a polynomial does not apply only on code-based cryptosystem, a search of methods which are not being used in code-based cryptography will be done, and we will list together with the methods indexed in the previous item;
    \item Propose countermeasures: Since the listed algorithm was not originally designed to blind a side-channel attacker, we will propose countermeasures in order to protect the implementation and avoid to leakage sensitive information;
    \item implementation: in order to effectively measure the efficacy of our proposal, we will implement the countermeasures presented to analyze the information leakage in our works.
    \item Analysis: We will use our implementation to measure the time variance of all studied methods, with and without the countermeasures proposed. Additionally, we will present an analysis of the execution time of each algorithm with different parameters;
    \item Summarize the obtained results: all results obtained will be summarized, and a conclusion will be drawn, defining the results obtained and the future works.
\end{enumerate}

\section{Scientific contribution}
The timing side-channel attack performed in Chapter~\ref{ch:code-based} and the countermeasures proposed in Chapter~\ref{ch:roots} results in the following paper:

\begin{itemize}
    \item  MARTINS, D.; BANEGAS, G.; CUSTÓDIO, R. Don’t Forget Your Roots: Constant-Time Root Finding over $\mathbb{F}_2^m$. In: International Conference on Cryptology and Information Security in Latin America.  2019. p. 109-129. Available in: \url{https://doi.org/10.1007/978-3-030-30530-7_6}.
\end{itemize}

This dissertation includes direct text from this publication.

\section{Organization}
The remainder of this thesis is organized as follows. In Chapter~\ref{ch:math}, we give a brief description of the mathematical background to better understanding this works. The McEliece cryptosystem, BIGQUAKE submission, and the timing side-channel attack are presented in Chapter~\ref{ch:code-based}. In Chapter~\ref{ch:roots}, the core of this thesis, we present five methods for finding roots over $\mathbb{F}_{2^m}$. We also include countermeasures for avoiding timing attacks. Chapter~\ref{ch:comparison} provides a comparison of the number of cycles of the original implementation and the implementation with countermeasures, a performance analysis, and the security consideration of our proposal. At last, in Chapter~\ref{ch:final}, we conclude this thesis and discuss open problems.
