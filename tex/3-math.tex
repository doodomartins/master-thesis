This chapter explain the necessary background to understand this works. A brief mathematical and coding theory concepts, followed by a explanation of cryptography primitives used in this works and finally, a explanation of side-channel attacks.

\section{Algebraic structures}
\subsection{Finite fields}
Explain main proprieties, main operations
\subsection{Finite field extensions}

\subsection{Polynomials over finite fields}
\section{Coding theory}
\subsection{Linear codes}
\subsection{Goppa codes}
Let $m, n, t\in \mathbb{N}$. A binary Goppa code $\Gamma(L, g(z))$ is defined by a polynomial $g(z) = \sum_{i=0}^{t}g_iz^i$ over $\mathbb{F}_{2^m}$ with degree $t$ and supported by $L = (\alpha_1, \alpha_2, \dots, \alpha_n) \in \mathbb{F}_{2^m}$ with $\alpha_i \neq \alpha_j$ for $i\neq j$, such that $g(\alpha_i) \neq 0$ for all $\alpha_i \in L$ and $g(z)$ is square free. To a vector  $c = (c_1 \ldots, c_{n}) \in \mathbb{F}^n_{2}$ we associate a syndrome polynomial associate the syndrome polynomial
\begin{align}
  S_c(z) = \sum_{i=1}^{n} \frac{c_i}{z+\alpha_i},  
\end{align}
where ${z+\alpha_i}$ is invertible $\pmod{g(z)}$, i.e., $(z+\alpha_i) \times \frac{1}{z+\alpha_i} \equiv 1 \pmod{g(z)}$.
\begin{definition}
The binary Goppa code $\Gamma(L, g(z))$ consists of all vectors $c \in \mathbb{F}_{2}^n$ such that
\begin{equation}
    S_c(z) \equiv 0 \bmod{g(z)}.
\end{equation}
\end{definition}

% Change this
The parameters of a linear code are the size $n$, dimension $k$ and minimum distance $d$. We use the notation $[n,k,d]-$Goppa code for a binary Goppa code with parameters $n,k$ and $d$. If the polynomial $g(z)$ which defines a Goppa code is irreducible over $\mathbb{F}_{2^m}$, we call the code an irreducible Goppa code.

The length of a Goppa code is given by $n = |L|$ and its dimension is $k \geq n-mt$, where $t = deg(g)$, and the minimum distance of $\Gamma(L, g(z))$ is $d \geq 2t + 1$. The syndrome polynomial $S_c(z)$ can be written as:
\begin{equation}
    S_c(z) \equiv \frac{w(z)}{\sigma(z)} \mod g(z),
\end{equation}
where $\sigma(z) = {\displaystyle \prod_{i=1}^{l}(z+\alpha_i)}$ is the product over those $(z+\alpha_i)$ where there is an error in position $i$ of $c$. This polynomial $\sigma(z)$ is called Error-Locator Polynomial (ELP).

A binary Goppa code can correct a codeword $c \in \mathbb{F}_{2}^n$, obscured by an error vector $e \in \mathbb{F}_{2}^n$ with Hamming weight $w_h(e)$ up to $t$, i.e., the numbers of non-zero entries in $e$ is at most $t$. The way to correct errors is using a decoding algorithm. For irreducible binary Goppa codes we have three alternatives for that. The extended Euclidean Algorithm (EEA) and Berlekamp-Massey algorithm are out of the scope of this work, because they needed a parity-check matrix that has twice more rows then columns. The Patterson algorithm~\cite{patterson1975algebraic}, focus of this paper, can correct up to $t$ errors with a smaller structure.
\section{Cryptography primitives}
\subsection{Public-key cryptography}
\subsection{Key encapsulation mechanisms}
\section{Side-channel attacks}
\subsection{Timing side-channel attacks}
