
\begin{lstlisting}[caption={Code snippet of BIGQUAKE error attribution },label={lst:error_original},language=C]
//Construct an error e from m
int error = (int *) malloc(NB_ERRORS*sizeof(int));
m2errir(m, error);

//Encrypt e (Niederreiter)
unsigned char* syndrome = 
  malloc(SYNDROME_BYTES*sizeof(unsigned char));
decrypt_nied(syndrome, error, (unsigned char*) sk);
\end{lstlisting}

\begin{lstlisting}[caption={Code snippet of BIGQUAKE error attribution with the fix. },label={lst:error_fix},language=C]
//Construct an error e from m
int error = {0};
m2errir(m, error);

//Encrypt e (Niederreiter)
unsigned char* syndrome = 
  malloc(SYNDROME_BYTES*sizeof(unsigned char));
decrypt_nied(syndrome, error, (unsigned char*) sk);
\end{lstlisting}

\begin{lstlisting}[caption={Multiplication of two elements in $\mathbb{F}_{2^{12}}$ and inversion of an element in $\mathbb{F}_{2^{12}}$},label={lst:mult},language=C]
#include <stdint.h>
typedef uint16_t gf;

//Multiplication between in0 and in1
gf gf_q_m_mult(gf in0, gf in1) {
    uint64_t i, tmp, t0 = in0, t1 = in1;
    //Multiplication
    tmp = t0 * (t1 & 1);
    for (i = 1; i < 12; i++)
        tmp ^= (t0 * (t1 & (1 << i)));
    //reduction
    tmp = tmp & 0x7FFFFF;
    //first step of reduction
    gf reduction = (tmp >> 12);
    tmp = tmp & 0xFFF;
    tmp = tmp ^ (reduction << 6);
    tmp = tmp ^ (reduction << 4);
    tmp = tmp ^ reduction << 1;
    tmp = tmp ^ reduction;
    //second step of reduction
    reduction = (tmp >> 12);
    tmp = tmp ^ (reduction << 6);
    tmp = tmp ^ (reduction << 4);
    tmp = tmp ^ reduction << 1;
    tmp = tmp ^ reduction;
    tmp = tmp & 0xFFF;
    return tmp;
}
\end{lstlisting}

\begin{lstlisting}[caption={Inversion of an element in $\mathbb{F}_{2^{12}}$},label={lst:mult-inv},language=C]
#include <stdint.h>
typedef uint16_t gf;

// Multiplicative inverse of in
gf gf_inv(gf in) {
    gf tmp_11 = 0;
    gf tmp_1111 = 0;
    gf out = in;
    out = gf_sq(out); //a^2
    tmp_11 = gf_mult(out, in); //a^2*a = a^3
    out = gf_sq(tmp_11); //(a^3)^2 = a^6
    out = gf_sq(out); // (a^6)^2 = a^12
    tmp_1111 = gf_mult(out, tmp_11); //a^12*a^3 = a^15
    out = gf_sq(tmp_1111); //(a^15)^2 = a^30
    out = gf_sq(out); //(a^30)^2 = a^60
    out = gf_sq(out); //(a^60)^2 = a^120
    out = gf_sq(out); //(a^120)^2 = a^240
    out = gf_mult(out, tmp_1111); //a^240*a^15 = a^255
    out = gf_sq(out); // (a^255)^2 = 510
    out = gf_sq(out); //(a^510)^2 =  1020
    out = gf_mult(out, tmp_11); //a^1020*a^3 = 1023
    out = gf_sq(out); //(a^1023)^2 = 2046
    out = gf_mult(out, in); //a^2046*a = 2047
    out = gf_sq(out); //(a^2047)^2 = 4094
    return out;
}
\end{lstlisting}