As argued, the leading cause of information leakage in the decoding algorithm is the process of finding the roots of the ELP. In general, the time needed to find these roots varies, often depending on the roots themselves. Thus, an attacker who has access to the decoding time can infer these roots, and hence get the vector of errors $e$. One example of this attack are presented in Section~\ref{sec:attack}, where we perform an side-channel attack over the decryption part of a cryptosystem and we recover the secret exchange between two parts of the protocol.

Factoring a polynomial it is a well studied problem on finite fields area. More recently, with the rising of the code-based area, some works starts to deal with the factorization problem on the decoding process in order to avoid timing side-channel attacks~\cite{shoufan2009timing, bucerzan2017improved}. In our systematic review, presented on Chapter~\ref{ch:method}, we select five methods used on code-based cryptosystem to analyse and to propose modifications in theirs implementation to made then available on cryptosystems applications.

The adjustments were made in the following algorithms to find roots: exhaustive search, linearized polynomials, Berlekamp trace algorithm (BTA), and successive resultant algorithm (SRA). Also, we present the Cantor-Zassenhaus method, as it has a probabilistic algorithm that made use of an random selection of polynomials in their execution, we made an security analysis over this algorithm to evaluate their blindness on a side-channel attack scenario. The description of each algorithm are presented in the next sections and the analysis are presented on Chapter~\ref{ch:comparison}.

As the main focus of this works relies on the root computation of an polynomial over a finite field, we restrict our scope to polynomials over binary fields. Also, we assume that we want to compute the roots of a squarefree polynomial, see~\cite{von2001factoring} for achieve squarefree factorization.

\section{Exhaustive search}
The exhaustive search, also called Chien Search~\cite{chien1964cyclic}, is a direct method, in which the evaluation of $f$ for all the elements in $\mathbb{F}_{2^m}$ is performed. A root is found whenever the evaluation result is zero. This method is acceptable for small fields and can be made efficient with a parallel implementation. The greatest drawback of this method it was the complexity of this algorithm, and this is the reason because their are not widely used. However, we consider that their are still relevant, it is because some NIST proposals are based on a small finite field, enabling their usage. 

Algorithm~\ref{alg:exhaustive} describes the exhaustive search. For every element in $\mathbb{F}_{2^m}$ we evaluate the polynomial and check if it was a root or not, and this method leaks information when we found a root. This is because whenever we found, i.e., $dummy = 0$, an extra operation is performed by adding the root founded on the returning list $R$. In this way, the attacker can infer from this additional time that a root was found, thus providing ways to obtain data that should be secret.

\begin{figure}[ht]
\begin{algorithm}[H]
 \KwData{$p(x)$ as univariate polynomial over $\mathbb{F}_{2^m}$ with $d$ roots, $A = [a_0, \ldots, a_{n-1}]$ as all elements in $\mathbb{F}_{2^m}$, $n$ as the length of $A$.}
 \KwResult{$R$ as a set of roots of $p(x)$.}
 $R \gets \emptyset$\;
\For{$i\gets0$ \KwTo $n-1$}{
    $dummy \gets p(A[i])$\;
   \If{$dummy == 0$}{
        $R.add(A[i])$\;
    }
}
\Return $R$\;
  \caption{Exhaustive search algorithm for finding roots of a univariate polynomial over $\mathbb{F}_{2^m}$.}\label{alg:exhaustive}
\end{algorithm}
\end{figure}

One solution to avoid this leakage is to permute the elements of vector $A$ before start the evaluation. Using this technique, an attacker can identify the extra operation, but without learning any secret information. In our case, we use the Fisher-Yates shuffle~\cite{black2005fisher} for shuffling the elements of vector $A$. In~\cite{wang2018fpga}, the authors show an implementation of the shuffling algorithm safe against timing attacks, thus we can build and exhaustive search without leak information against a side-channel attack. Algorithm~\ref{alg:exhaustive_permuted} shows the permutation of the elements and the computation of the roots.

\begin{figure}[ht]
\begin{algorithm}[H]
 \KwData{$p(x)$ as a univariate polynomial over $\mathbb{F}_{2^m}$ with $d$ roots, $A = [a_0, \ldots, a_{n-1}]$ as all elements in $\mathbb{F}_{2^m}$, $n$ as the length of $A$.}
 \KwResult{$R$ as a set of roots of $p(x)$.}
  permute$(A)$\;
 $R \gets \emptyset$\;
\For{$i\gets0$ \KwTo $n-1$}{
    $dummy \gets p(A[i])$\;
   \If{$dummy == 0$}{
        $R.add(A[i])$\;
    }
}
\Return $R$\;
 \caption{Exhaustive search algorithm with a countermeasure for finding roots of an univariate polynomial over $\mathbb{F}_{2^m}$.}
  \label{alg:exhaustive_permuted}
\end{algorithm}
\end{figure}

Using this approach, we add one extra step to the algorithm. However, this permutation blurs the sensitive information of the algorithm, making the usage of Algorithm~\ref{alg:exhaustive_permuted} slightly harder for the attacker to acquire timing leakage. For this case, we want to avoid the addition of an \texttt{else} clause adding the elements that are not a root in other list. This addition will absolutely reduce the time variance on the execution of the algorithm, but it could open a lack for cache attacks~\cite{dan2005}.


\section{Berlekamp Trace Algorithm}
Our second strategy was the classical Berlekamp factoring algorithm~\cite{berlekamp1970factoring}. Berlekamp presents an efficient algorithm to factor a polynomial, which can be used to find its roots. We call this algorithm \emph{Berlekamp trace algorithm} since it is relies on the trace function properties to split a polynomial into two small polynomials. This classical algorithm is one of the most used on code-based cryptosystems for the reason that the Berlekamp has lower complexity when compared with exhaustive search~\cite{biswas2009efficient}. 

The trace function is defined as $Tr(x) = x + x^{2} + x^{2^{2}} + \dots + x^{2^{m-1}}$. It is possible to change BTA for finding roots of a polynomial $p(x)$ using $\beta = \{\beta_1, \beta_2, \ldots, \beta_m\}$ as a standard basis of $\mathbb{F}_{2^{m}}$, and then computing the greatest common divisor between $p(x)$ and $Tr(\beta_0 \cdot x)$. After that, it starts a recursion where BTA performs two recursive calls; one with the result of gcd algorithm and the other with the remainder of the division $p(x) / \gcd(p(x), Tr(\beta_i \cdot x))$. On the next call, the BTA must use a different basis then the previous call. 

The base case is when the degree of the input polynomial is smaller than one. In this case, BTA returns the root, by getting the independent term of the polynomial. In summary, the BTA is a divide and conquer like algorithm since it splits the task of computing the roots of a polynomial $p(x)$ into the roots of two smalls polynomials. Some improvements can be made on BTA with a hybrid version, i. e. when the degree of the polynomial are equals two, we can use a different algorithm to find the factors of then, as presented in~\cite{strenzke2012fast}. As efficiency was not the focus of this work, we don't consider this approach. All steps of the Berlekamp trace algorithm are describe in Algorithm~\ref{alg:bta}.


\begin{figure}[ht]
\begin{algorithm}[H]
 \KwData{$p(x)$ as a univariate polynomial over $\mathbb{F}_{2^m}$ and i.}
 \KwResult{The set of roots of $p(x)$.}
    \If{$deg(p(x)) \leq 1$}{
        \Return root of $p(x)$\;
    }
    $p_{0}(x) \gets gcd(p(x), Tr(\beta_{i}\cdot x))$\;
    $p_{1}(x) \gets p(x) / p_{0}(x)$ \;
\Return $BTA(p_{0}(x), i + 1) \cup BTA(p_{1}(x), i + 1)$\;
 \caption{Berlekamp Trace Algorithm -- $BTA(p(x), i)-rf$.}
  \label{alg:bta}
\end{algorithm}
\end{figure}

As we can see, a direct implementation of Algorithm~\ref{alg:bta} has no constant execution time. The recursive behavior may leak information about the characteristics of roots in a side-channel attack. Additionally, in our experiments, we noted that the behavior of the gcd with the trace function may result in a polynomial with the same degree. Therefore, BTA will divide this input polynomial in a future call with a different basis. Consequently, there is no guarantee of a constant number of executions because we cant control if an polynomial will be split or not. 

In order to avoid the nonconstant number of executions and avoid timing side-channel attacks, here referred as $BTA-it$, we propose an iterative implementation of Algorithm~\ref{alg:bta}. In this way, our proposal iterates in a fixed number of iterations instead of calling itself until the base case. The main idea is not changed; we still divide the task of computing the roots of a polynomial $p(x)$ into two smaller instances. However, we change the approach of the division of the polynomial. Since we want to compute the same number of operations independent of the degree of the polynomial, we perform the gcd with a trace function for all basis in $\beta$, and choose a division that results in two new polynomials with approximate degree.

This new approach allows us to define a fixed number of iterations for our version of BTA. Since we always divide into two small instances, we need $t-1$ iterations to split a polynomial of degree $t$ in $t$ polynomials of degree $1$.
Then we just need to add an stack to controlling the polynomials to be divided.
Algorithm~\ref{alg:ibta} presents our new approach of the iterative BTA -- $i-BTA$.

\begin{figure}[ht]
\begin{algorithm}[H]
 \KwData{$p(x)$ as an univariate polynomial over $\mathbb{F}_{2^m}$, $t$ as number of expected roots.}
 \KwResult{The set of roots of $p(x)$.}
    $g \gets \{p(x)\};$ \tcp{The set of polynomials to be computed}
    \For{$k \gets 0$ \KwTo $t$}{
        $current = g.pop()$\;
        Compute $candidates$ $=$ $gcd(current, Tr(\beta_{i}\cdot x))$ $\forall$ $\beta_{i}$ $\in$ $\beta$\;
        Select $p_{0}$ $\in$ $candidates$ such as $p_{0}.degree$ $\simeq$ $\frac{current}{2}$\;
        $p_{1}(x) \gets current / p_{0}(x)$ \;
        \If{$p_{0}.degree == 1$}{
            $R.add($root of $p_{0})$
        } \Else{
            $g.add(p_{0})$\;
        }
        \If{$p_{1}.degree == 1$}{
            $R.add($root of $p_{1})$
        } \Else{
            $g.add(p_{1})$\;
        }
    }
    \Return{$R\;$}
 \caption{Iterative Berlekamp Trace Algorithm -- $BTA(p(x))-it$.}
  \label{alg:ibta}
\end{algorithm}
\end{figure}
Algorithm~\ref{alg:ibta} extracts a root of the polynomial when the variable $current$ has a polynomial with degree equal to one. If this degree is greater than one, then the algorithm needs to continue dividing the polynomial until it finds a root. The algorithm does that by adding the polynomial in a stack and reusing this polynomial in a future division. The iterative BTA has a higher complexity when compare to the original BTA. Nevertheless, with our new approach, we achieve a more constant time execution on the root finding task, consequently reducing the knowledge obtained by an attacker. On Chapter~\ref{ch:comparison} we analyse and compare our iterative BTA with the others approach.

\section{Linearized Polynomials}
The third countermeasure proposed is based on linearized polynomials. The authors in \cite{fedorenko2002finding} propose a method to compute the roots of a polynomial over $\mathbb{F}_{2^m}$, using a particular class of polynomials, called linearized polynomials. In \cite{strenzke2012fast}, this approach is a recursive algorithm which the author calls ``dcmp-rf''. In our solution, however, we present an iterative algorithm. First, we define linearized polynomials as follows:

\begin{definition}
A polynomial $\ell(y)$ over $\mathbb{F}_{2^m}$ is a linearized polynomial if
\begin{equation}
    \ell(y) = \sum_i c_iy^{2^i},
\end{equation}
where $c_i \in \mathbb{F}_{2^m}$.
\end{definition}
In addition, from~\cite{truong2001fast}, we have Lemma~\ref{lemma:lin} that describes the main property of linearized polynomials for finding roots.
\begin{lemma}
\label{lemma:lin}
    Let $y \in \mathbb{F}_{2^m}$ and let $\alpha^0, \alpha^1, \ldots, \alpha^{m-1}$ be a standard basis over $\mathbb{F}_2$. If
    \begin{equation}
        y = \sum_{k=0}^{m-1} y_k\alpha^k,\quad y_k \in \mathbb{F}_2
    \end{equation}
    and $\ell(y) =\sum_j c_jy^{2^j}$, then
      \begin{equation}
        \ell(y) = \sum_{k=0}^{m-1} y_k\ell(\alpha^k).
    \end{equation}
\end{lemma}

We call $A(y)$ over $\mathbb{F}_{2^m}$ an affine polynomial if $A(y) = \ell(y) + \beta$ for $\beta \in \mathbb{F}_{2^m}$, where $\ell(y)$ is a linearized polynomial. Using this definitions, we can construct a method for finding the ELP roots. First, we present a toy example from~\cite{truong2001fast} to understand the idea behind finding roots using linearized polynomials.

\begin{example}\label{ex:1}
Let us consider the polynomial $f(y) = y^2 + (\alpha^2+1)y + (\alpha^2 +\alpha +1)y^0$ over $\mathbb{F}_{2^3}$ and $\alpha$ a primitive element in $\mathbb{F}_2[x]/ x^3+x^2+1$. Since we are trying to find roots, we can write $f(y)$ equals to zero
 $$ y^2 + (\alpha^2+1)y + (\alpha^2 +\alpha +1)y^0 = 0$$
 and after that, we can rewrite the sentence like
\begin{equation}\label{eq:example_1}
    y^2 + (\alpha^2+1)y   = (\alpha^2 +\alpha +1)y^0.
\end{equation}
We can point that on the left hand side of Equation~\ref{eq:example_1}, $\ell(y) = y^2 + (\alpha^2+1)y$ is a linearized polynomial over $\mathbb{F}_{2^3}$ and Equation~\ref{eq:example_1} can be expressed just as
\begin{equation}\label{eq:example_1_2}
    \ell(y) = \alpha^2 +\alpha +1
\end{equation}
If $y = y_2\alpha^2 + y_1\alpha + y_0 \in \mathbb{F}_{2^3}$ then, according to Lemma~\ref{lemma:lin}, Equation~\ref{eq:example_1_2} becomes
\begin{equation}\label{eq:example_1_3}
    y_2\ell(\alpha^2) + y_1\ell(\alpha) + y_0\ell(\alpha^0) = \alpha^2 +\alpha +1
\end{equation}
We can compute $\ell(\alpha^0),\ell(\alpha)$ and $\ell(\alpha^2)$ using the left hand side of Equation~\ref{eq:example_1} and we have the following values
\begin{equation}\label{eq:example_1_4}
    \begin{split}
        \ell(\alpha^0) & = (\alpha^0)^2 + (\alpha^2+1)(\alpha^0) = \alpha^2+1 + 1 = \alpha^2, \\
        \ell(\alpha) & = (\alpha)^2 + (\alpha^2+1)(\alpha) = \alpha^2 + \alpha^2+ \alpha + 1 = \alpha + 1, \\
        \ell(\alpha^2) & = (\alpha^2)^2 + (\alpha^2+1)(\alpha^2) = \alpha^4 +\alpha^4 +  \alpha^2 = \alpha^2.
    \end{split}
\end{equation}
A substitution of Equation~\ref{eq:example_1_4} into Equation~\ref{eq:example_1_3} gives us
\begin{equation}\label{eq:example_1_5}
     (y_2+y_0)\alpha^2 + (y_1)\alpha + (y_1)\alpha^0 = \alpha^2 +\alpha +1
\end{equation}
Equation~\ref{eq:example_1_5} can be expressed as a matrix in the form
\begin{equation}\label{eq:example_1_6}
    \begin{bmatrix} y_2 & y_1 & y_0 \end{bmatrix}
    \begin{bmatrix}
    1 & 0 & 0 \\
    0 & 1 & 1 \\
    1 & 0 & 0
    \end{bmatrix}
    =
    \begin{bmatrix} 1 & 1 & 1 \end{bmatrix}.
\end{equation}
If one solves simultaneously the linear system in Equation~\ref{eq:example_1_6} then the results are the roots of the original polynomial given in Equation~\ref{eq:example_1}. From Equation~\ref{eq:example_1_5}, one observes that the solutions are $y=110$ and $y=011$. Furthermore, we can translate $110$ and $011$ to $\alpha + 1$ and $\alpha^2 + \alpha$. After all this steps, is easy to check if we found the factors of $f(y)$, we just need to multiply $y - \alpha + 1$ and $y- \alpha^2 + \alpha$, thus we get

\begin{equation*}\label{eq:example_1_7}
    (y - \alpha + 1) (y - \alpha^2 + \alpha) = y^2 + (\alpha^2+1)y + (\alpha^2 +\alpha +1) = f(y).
\end{equation*}
\end{example}

Fortunately, the authors in \cite{fedorenko2002finding} provide a generic decomposition for finding affine polynomials. Thus we can transform the error locator polynomial in an affine polynomial. In their work, each polynomial in the form $F(y) = \sum_{j=0}^{t} f_jy^j$ for $f_j \in \mathbb{F}_{2^m}$ can be represented as
\begin{equation}
\label{eq:f_y}
    F(y) = f_3y^3 + \sum_{i=0}^{\lceil (t-4)/5 \rceil} y^{5i}(f_{5i} + \sum_{j=0}^{3} f_{5i+2^j}y^{2^j}).
\end{equation}

After that, we can summarize all the steps on Algorithm~\ref{alg:linearized}. The function ``generate($m$)'' refers to the generation of the elements in $\mathbb{F}_{2^m}$ using Gray codes, see \cite{savage1997survey} for more details about Gray codes. The linearized method differ from the exhaustive search on the evaluation process, here, it is much more efficiency compute this evaluation because of the polynomial are in an linearized form. Algorithm~\ref{alg:linearized} presents our countermeasure in the last steps of the algorithm, i.e., we added a dummy operation for blinding if $X[j]$ is a root of polynomial $F(x)$. The analysis of linearized method and its countermeasure are presented in Chapter~\ref{ch:comparison}.

\begin{figure}
\begin{algorithm}[H]
 \KwData{$F(x)$ as a univariate polynomial over $\mathbb{F}_{2^m}$ with degree $t$ and $m$ as the extension field degree.}
 \KwResult{$R$ as a set of roots of $p(x)$.}
 $\ell^k_i \gets \emptyset$; $\ell_{is} \gets \emptyset$; $A^j_k \gets \emptyset$; $R \gets \emptyset$; $dummy \gets \emptyset$\;
 \If{$f_0  == 0$}{
 $R.append(0)$\;
 }
 \For{$i\leftarrow 0$ \KwTo $\lceil (t-4)/5 \rceil$}
 {
    $\ell_i(x) \gets 0$\;
    \For{$j\gets 0$ \KwTo $3$}{
      $\ell_i(x) \gets \ell_i(x) + f_{5i+2^j}x^{2^j}$\;
      }
    $\ell_{is}[i] \gets \ell_i(x)$\;
 }
 \For{$k\gets 0$ \KwTo $m-1$}{
    \For{$i\leftarrow 0$ \KwTo $\lceil (t-4)/5 \rceil$}
    {
        $\ell^k_i \gets \ell_{is}(\alpha^k)$\;
    }

 }
 $A^0_i \gets \emptyset$\;
 \For{$i\gets 0$ \KwTo $\lceil (t-4)/5 \rceil$}{
  $A^0_i \gets f_{5i}$\;
 }

 $X \gets \text{generate}(m)$\;
 \For{$j\gets 1$ \KwTo $2^m - 1$}{
    \For{$i\gets 0$ \KwTo $\lceil (t-4)/5 \rceil$}{
        $A \gets A^{j-1}_i$\;
        $A \gets A + \ell^{\delta(X[j], X[j-1])}_i$\;
        $A^j_i \gets A$\;
    }
 }
\For{$j\gets 1$ \KwTo $2^m - 1$}{
    $result \gets 0$\;
    \For{$i\gets 0$ \KwTo $\lceil (t-4)/5 \rceil$}{
        $result = result + (X[j])^{5i}A^j_i$\;
    }
    $eval = result + f_3(X[j])^{3}$\;
    \eIf{$eval == 0$}{
        $R.append(X[j])$\;
    }{
        $dummy.append(X[j])$\;
    }
}
\Return $R$\;
 \caption{Linearized polynomials for finding roots over $\mathbb{F}_{2^m}$.}
  \label{alg:linearized}
\end{algorithm}
\end{figure}

\section{Successive Resultant Algorithm}
In \cite{petit2014finding}, the authors present an alternative method for finding roots in $\mathbb{F}_{p^m}$. Later on, the authors better explain the method in~\cite{petit2016finding}. The Successive Resultant Algorithm (SRA) relies on the fact that it is possible to find roots exploiting properties of an ordered set of rational mappings.

Given a polynomial $f$ of degree $d$ and a sequence of rational maps $K_1,\ldots, K_t$, the algorithm computes finite sequences of length $j \leq t+1$ obtained by successively transforming the roots of $f$ by applying the rational maps. The algorithm is as follows: Let $\{v_1,\ldots,v_m\}$ be an arbitrary basis of $\mathbb{F}_{p^m}$ over $\mathbb{F}_p$, then it is possible to define $m+1$ functions $\ell_0, \ell_1,\ldots, \ell_m$ from $\mathbb{F}_{p^m}$ to $\mathbb{F}_{p^m}$ such that
$$
\left \{
\begin{array}{l}
     \ell_0(z) = z\\
     \ell_1(z) = \prod_{i\in \mathbb{F}_p}\ell_0(z-iv_1)\\
     \ell_2(z) = \prod_{i\in \mathbb{F}_p}\ell_1(z-iv_2)\\
     \vdots \\
     \ell_m(z) = \prod_{i\in \mathbb{F}_p}\ell_{m-1}(z-iv_m)\\
\end{array}
\right.
$$
The functions $\ell_j$ are examples of linearized polynomials, as previously defined. Our next step is to present the theorems from \cite{petit2014finding}. Check original work for the proofs.
\begin{theorem}\label{lemma_1}
\begin{itemize}
    \item[a)] Each polynomial $\ell_i$ is split and its roots are all elements of the vector space generated by $\{v_1, \ldots,v_i\}$. In particular, we have $\ell_n(z) = z^{p^m} -z$.
    \item[b)] We have $\ell_i(z)  = \ell_{i-1}(z)^p - a_i\ell_{i-1}(z)$ where $a := (\ell_{i-1}(v_i))^{p-1}$.
    \item[c)] If we identify $\mathbb{F}_{p^m}$ with the vector space $(\mathbb{F}_p)^m$, then each $\ell_i$ is a $p$-to-$1$ linear map of $\ell_{i-1}(z)$ and a $p^i$ to $1$ linear map of $z$.
\end{itemize}
\end{theorem}

From Theorem~\ref{lemma_1} and its properties, we can reach the following polynomial system:
\begin{equation}\label{eq:system_1}
    \left \{
\begin{array}{l}
    f(x_1) = 0\\
     x_j^p = a_jx_j = x_{j+1} \quad j=1,\ldots, m-1\\
     x_n^p - a_nx_n = 0
\end{array}
\right.
\end{equation}
where the $a_i \in \mathbb{F}_{p^n}$ are defined as in Theorem~\ref{lemma_1}. Any solution of this system provides us with a root of $f$ by the first equation, and the $n$ last equations together imply this root belongs to $\mathbb{F}_{p^n}$. From this system of equations,~\cite{petit2014finding} derives Theorem~\ref{lemma_2}.

\begin{theorem}\label{lemma_2}
Let $(x_1,x_2,\ldots,x_m)$ be a solution of the equations in Equation~\ref{eq:system_1}. Then $x_1 \in \mathbb{F}_{p^m}$ is a solution of $f$. Conversely, given a solution $x_1 \in \mathbb{F}_{p^m}$ of $f$, we can reconstruct a solution of all equations in Equation~\ref{eq:system_1} by setting $x_2 =x_1^p - a_1x_1$, etc.
\end{theorem}

In \cite{petit2014finding}, the authors present an algorithm for solving the system in Equation~\ref{eq:system_1} using resultants. The solutions of the system are the roots of polynomial $f(x)$. It is worth remarking that this algorithm is almost constant-time and hence we just need to protect the branches presented on it.


\section{Rabin Algorithm}
Our last proposal for finding roots of the error locator polynomial was the classical method proposed by Rabin~\cite{rabin1980probabilistic}. This classical method is an probabilistic algorithm and has running time similar to the Cantor-Zassenhaus algorithm~\cite{cantor1981new}. Their are used to factor huge polynomials, with higher degree and over larger fields. Although the main focus of this works relies on polynomials with no more then 200 roots and fields of size at most $2^{18}$, we consider this algorithm because of its efficiency.

The Cantor-Zassenhaus differ from Rabin's method on the field extension of the polynomial to be factored. Since the Cantor are designer only for odd extensions, we does not consider to this works. However, Rabin made and adaptation to support even fields, introducing an trace computation to finding nontrivial factorization of the input polynomial, similar to the BTA method. This addition increases the execution time of the algorithm, but it steel is efficiency, even when it is used on large fields. 

The main difference between Cantor-Zassenhaus and Rabin methods to the others methods presented in this chapter was the insertion of an randomness choice of an element in the algorithm. The computation of the roots depends was given by the choice of an random element in $\mathbb{F}_{2^{m}}$, this random selection was used with an trace function and an gcd to find an nontrivial factorization of the original polynomial. This class of algorithm are also called probabilistic algorithms.

Cantor-Zassenhaus is an probabilistic algorithm because this random behaviour, and the parameter $\epsilon$ was used to define the maximum number of iterations on the algorithm. The probability of failure of the algorithm was on the fact that a random selection could not result in a nontrivial factor of the input polynomial. However, Ben-Or prove that this probability is $1 - 1/2^{t-1}$, where $t$ are the degree of the polynomial~\cite{ben1981probabilistic}.

To the best of our knowledge, any code-based cryptosystem made uses of an probabilistic algorithm as method for root finding. The original proposal was designed as an recursive algorithm, here we present an iterative version, as presented in~\cite{von2001factoring}. Algorithm \ref{alg:new} shows all steps to compute the roots. We do not propose any change on the Rabin's algorithm, since it uses an random selection on their execution, we consider that this randomness does not permit an attacker to infer any information about the roots. Moreover, since we are using a random selection, if an attacker could not infer any information from the time variance of the algorithm. A more detailed analysis was presented in Chapter~\ref{ch:comparison}.

\begin{figure}[ht]
\begin{algorithm}[H]
 \KwData{$f(x)$ as an univariate polynomial over $\mathbb{F}_{2^m}$, $t$ as number of expected roots.}
 \KwResult{The set of roots of $p(x)$.}
    $Factors \gets \{f\};$ \tcp{The set of polynomials to be computed}
    $k \gets 1$\;
    $it_{max} \gets 2 \lceil\log{\frac{t^2}{\epsilon}}\rceil$\;
    \While{$k < it_{max}$}{
        $h \xleftarrow{\$} \mathbb{F}_{2^m}$ \tcp{choose $h$ with degree $< t$ at random}
        $g \gets gcd(h, f)$\;
        \If{$g = 1$}{
            $g \gets Tr(h)$
        }
        \For{$fact \in Factors \quad$and$\quad fact.degree > d$}{
            \If{$p_{0}.degree == 1$}{
                $Factors.remove(u)$\;
                $Factors.insert(gcd(g,u))$\;
                $Factors.insert(u/gcd(g,u))$\;
            }
        }
        \If{$p_{1}.degree == 1$}{
            $R.add($root of $p_{1})$
        }
    }
    \Return{$Factors\;$}
 \caption{Probabilistic root finding -- $Rabin(p(x))$.}
  \label{alg:new}
\end{algorithm}
\end{figure}
