\section{Exhaustive search}
The exhaustive search is a direct method, in which the evaluation of $f$ for all the elements in $\mathbb{F}_{2^m}$ is performed. A root is found whenever the evaluation result is zero. This method is acceptable for small fields and can be made efficient with a parallel implementation. Algorithm~\ref{alg:exhaustive} describes this method.

As can be seen in Algorithm~\ref{alg:exhaustive}, this method leaks information. This is because whenever a root is found, i.e., $dummy = 0$, an extra operation is performed. In this way, the attacker can infer from this additional time that a root was found, thus providing ways to obtain data that should be secret.
\vspace{-0.5cm}
\begin{algorithm}[ht]
 \KwData{$p(x)$ as univariate polynomial over $\mathbb{F}_{2^m}$ with $d$ roots, $A = [a_0, \ldots, a_{n-1}]$ as all elements in $\mathbb{F}_{2^m}$, $n$ as the length of $A$.}
 \KwResult{$R$ as a set of roots of $p(x)$.}
 $R \gets \emptyset$\;
\For{$i\gets0$ \KwTo $n-1$}{
    $dummy \gets p(A[i])$\;
   \If{$dummy == 0$}{
        $R.add(A[i])$\;
    }
}
\Return $R$\;
  \caption{Exhaustive search algorithm for finding roots of a univariate polynomial over $\mathbb{F}_{2^m}$.}
  \label{alg:exhaustive}

\end{algorithm}

One solution to avoid this leakage is to permute the elements of vector $A$. Using this technique, an attacker can identify the extra operation, but without learning any secret information. In our case, we use the Fisher-Yates shuffle~\cite{black2005fisher} for shuffling the elements of vector $A$. In~\cite{wang2018fpga}, the authors show an implementation of the shuffling algorithm safe against timing attacks. Algorithm~\ref{alg:exhaustive_permuted} shows the permutation of the elements and the computation of the roots.

\begin{algorithm}[ht]\label{def:linearized:polynomial}
 \KwData{$p(x)$ as a univariate polynomial over $\mathbb{F}_{2^m}$ with $d$ roots, $A = [a_0, \ldots, a_{n-1}]$ as all elements in $\mathbb{F}_{2^m}$, $n$ as the length of $A$.}
 \KwResult{$R$ as a set of roots of $p(x)$.}
  permute$(A)$\;
 $R \gets \emptyset$\;
\For{$i\gets0$ \KwTo $n-1$}{
    $dummy \gets p(A[i])$\;
   \If{$dummy == 0$}{
        $R.add(A[i])$\;
    }
}
\Return $R$\;
 \caption{Exhaustive search algorithm with a countermeasure for finding roots of an univariate polynomial over $\mathbb{F}_{2^m}$.}
  \label{alg:exhaustive_permuted}
\end{algorithm}

Using this approach, we add one extra step to the algorithm. However, this permutation blurs the sensitive information of the algorithm, making the usage of Algorithm~\ref{alg:exhaustive_permuted} slightly harder for the attacker to acquire timing leakage.
\section{Berlekamp Trace Algorithm}
\section{Linearized Polynomials}
\section{Successive Resultant Algorithm}
\section{Cantor-Zassenhaus Algorithm}
