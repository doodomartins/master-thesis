\begin{resumo}[Resumo]
  Aqui deve ser inserido um resumo de 150 a 500 palavras (limitação de tamanho dada pela BU). A linguagem deve ser português e a hifenização já foi alterada. O resumo em português deve preceder o resumo em inglês, mesmo que o trabalho seja escrito em inglês. A BU também diz que deve ser usada a voz ativa e o discurso deve ser na 3ª pessoa. A estrutura do resumo pode ser similar a estrutura usada em artigos: Contexto -- Problema -- Estado da arte -- Solução proposta  -- Resultados.

  % Atenção! a BU exige separação através de ponto (.). Ela recomanda de 3 a 5 keywords
  \vspace{\baselineskip} 
  \textbf{Palavras-chave:} Palavra-chave. Ponto como separador. Bla.
\end{resumo}


\begin{resumo}[Resumo Estendido]
  \section*{Introdução} 
  A hifenização é alterada para \texttt{brazil}, mesmo para documentos em inglês. Descrever brevemente esses itens exigidos pela BU. Como a RN 95/CUn/2017 é mais recente e impõe outras regras a revelia de regimentos e regulamentos, é mais sábio obedecê-la. Lembre que esse resumo estendido deve term entre 2 e 5 páginas.
  
  \lipsum[1]
  \section*{Objetivos} 
  \lipsum[2]
  \section*{Metodologia} 
  \lipsum[3]
  \section*{Resultados e Discussão} 
  \lipsum[4]
  \section*{Considerações Finais} 
  \lipsum[5]

  \vspace{\baselineskip}  % Atenção! manter igual ao resumo
  \textbf{Palavras-chave:} Palavra-chave. Outra Palavra-chave composta. Bla.
\end{resumo}

\begin{abstract}
  Enlish version of the plain ``resumo'' above. Done with environment
  \texttt{abstract}. Hyphenization is automatically changed to english.

  \vspace{\baselineskip} 
  \textbf{Keywords:} Keyword. Another Compound Keyword. Bla.
\end{abstract}